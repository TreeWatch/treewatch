\section{Improve Skills of a Scrum Master}
	\subsection{Reason to Choose}
		In the project during the 7th semester, which is the Software Factory, I chose the position of a Scrum Master. To successfully fill in this position and being able to support my group I wanted to improve my knowledge of this position during that semester.
		My decision was made for the following learning goal because of this reason.

	\subsection{S.M.A.R.T.}
	\begin{SMART}
	    \item[Specific] As the Scrum Master in our project, I want to improve my skills for this task.
	    \item[Measurable] This competence can be measured using a small questionnaire, because my team members can tell the best if I have made myself well.
	    \item[Attainable] To improve my skills in this area, I can find lots of information on the Internet but also benefit from the experience of my group.
	    \item[Relevant] Scrum is a methodology in many software projects of small and large companies.
	    \item[Time-limited] The semester ends in January 2016 which is why I will have reached my goal by the end of January.
	\end{SMART}
	
	\subsection{S.T.A.R.R.}
	\begin{STARR}
	    \item[Situation] We are working in the development phase with Scrum. So it needs to exist a Scrum Master.
	    \item[Task] I want to try to be the Scrum Master in my group. And to measure it I will create a questionnaire which my group members had to fill out for every Sprint.
	    \item[Action] I need to learn a lot about a Scrum Master and what he is doing.
	    \item[Result] In the end I got the informations of my group that I have to learn a lot more about the doings of a Scrum Master.
	    
	    		The results of the questionnaire can be found at the end of this document.
	    \item[Reflection] That is why I can now understand that you can get a certificate for Scrum Master. If I got the chance to work with Scrum I will definitely try to get those certificate.
	\end{STARR}