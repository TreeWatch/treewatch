\section{Scrum Planning}
	The Scrum planning took place at the beginning of the project, which happened between the second and third week of September. During the planning the criteria and time course were set.
	
	The results of the planning were shown inside the following sub sections. Furthermore you will find all the User Stories, the Cumulative Flow and the working hours, consists of a complete overview and a detailed description, in the following sections.

	\subsection{Sprint Planning}


		Time-Box für die Implementation eines Increment of Potentially Shippable Functionality durch das Scrum-Team Länge: 6 Kalendertage mit 3 Tagen pro Woche
		Beginnt (Kick-Off) mit dem Sprint Planning Meeting
		Endet mit dem Sprint Review Meeting und anschließendem Sprint Retrospective Meeting
		Während des Sprints wird das Team nicht durch neue oder geänderte Anforderungen unterbrochen. Damit erreicht man Kontinuität und konzentriertes Arbeiten auf das vom Product Owner ausgegebene Sprint Goal hin.
		Nicht-Team-Mitglieder (insbesondere der Product Owner u.a. Stakeholders) stehen während des Sprints für Rückfragen zur Verfügung.

		\subsubsection{Sprint Planning Meeting}
		
		
			Timebox: 2 Hour

			Ablauf Sprint Planning Meeting:
				- Arbeitspaket -> Sprint Backlog für Sprint
				- Erstellung der Aufgaben zusammen mit Project owner, project manager und Main Developer
					. project owner gibt sprint goal vor und nennt wichtige items die jetzt entwickelt werden sollen. gefolgt von Abstimmung der genauen ziele zwischen Gruppe und owner
					. Erstellung und Zuteilung von task aus den requirements

		\subsubsection{Daily Scrum Meeting}
		
			Timebox: 10 Minutes
		
			Kein Richtiges Daily Scrum Meeting … Kurze Absprache über aktuellen Stand der Tasks und der heutigen Planung

		\subsubsection{Sprint Review Meeting}


			Timebox: 1 Hour

			Fand gleichzeitig mit dem Planning Meeting statt.