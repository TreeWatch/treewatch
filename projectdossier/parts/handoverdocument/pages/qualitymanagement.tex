\section{Quality management}
This chapter describes the necessary information needed to manage the project quality from the project planning to the delivery
to the customer. Within this document the quality policies, procedures, roles, responsibilities and authorities are defined.

At the highest level quality management involves planning, doing, checking, and acting to improve project quality standards. Project quality management is therefore split into three categories: quality planning, quality assurance and quality control.
\subsection{Organization and Responsibilities}
\begin{table}[htbp]
	\begin{tabular}{ p{4cm} p{3.5cm} p{6cm} }
		\textbf{Name} & \textbf{Role} & \textbf{Quality Responsibility} \\ \hline
		Max van der Linden & Project Manager & External communication, Auditing \\
		Martijn Bonajo & Configuration Manager & Infrastructure, source files, Software engineering documentation, auditing \\
		Ron Gebauer & Scrum Master & Scrum planning, auditing \\
		Rene Karoff & Quality Manager & Ensure use of quality guidelines, auditing \\
		Jan Kerkenhoff & Main Engineer & Main Code Master, auditing \\
	\end{tabular}
	\caption{Group roles\label{tab:GroupRoles}}
\end{table}

\subsection{Quality Planning}
Since this project is about programming a mobile business \gls{app}, it is highly important that the system has a minimum of bugs/errors. A good documentation is necessary too, not only because this \gls{app} will be developed by another SoFa group after our semester is over, but the customer should understand the system which is developed by us.
\subsubsection{Define Project Quality}
To ensure to quality of the written code it is mandatory to develop every module test-driven, therefore unit-testing is introduced. The written documents will be checked for its grammar, writing style and content.
\subsubsection{Measure Project Quality}
Most of the code of this \gls{app} was tested, but since we have a lot of \gls{gui} related code which we weren't able to run unit tests with, we were not able to achieve the former stated goal of a code coverage of 90\%.

Furthermore the GUI was tested using the Xamarin Testcloud which was included in our Academic license which represents the Xamarin Business license. Since we had the Business license, we were allowed to use the Xamarin Testcloud for one hour per month. But since we had 5 licenses this expanded to 5 hours per month.

In table \ref{tab:documentReview} it is described who reviewed which document and with what rank it was approved.
\begin{sidewaystable}
	\begin{tabular}[htbp]{ p{7cm} *{5}{l} }
		\textbf{Document name} & \textbf{Author} & \textbf{Reviewer 1} & \textbf{Rank 1} & \textbf{Reviewer 2} & \textbf{Rank 2} \\ \hline
		User Stories & Max van der Linden & René Karoff & ++ & Martijn Bonajo & + \\
		Personal development plan
		Personal competence plan
		Mockup Design & Group work & René Karoff & ++ & Max van der Linden & ++ \\
		Functional Prototype & Handover &  &  &  &  \\
	\end{tabular}
	\captionof{table}{Document Review\label{tab:documentReview}}
\end{sidewaystable}

The code of the \gls{app} will be tested. Therefore it is a goal to get at least 90\% code coverage, better 100\%. Each evaluation criterion of the written documents will be ranked from “--“ to “++” (“--“ too bad – “++” excellent).

\subsection{Quality Assurance}
Since the code is tested, this will show the group if the quality goal is achieved. Furthermore the written documents will be audited by at least one group member and the project quality manager.
\subsubsection{Analyze Project Quality}
The tool to measure the code coverage \textbf{(which still needs to be defined)} will show the programmer which code is still uncovered, and the writer of a document will receive feedback of the auditing persons, so he can improve on his writing too.
\subsubsection{Improve Project Quality}
As stated in the project management plan, proper requirements were created to increase the project quality. This was done during two sprints, where the customer received the user stories, prioritized them and gave feedback to the group in the first sprint. During the 2nd sprint the user stories were reworked due to the remarks the customer gave to the group.

\subsection{Quality Control}
At the end of the project each deliverable will be audited again, by every team member to ensure the project quality.
