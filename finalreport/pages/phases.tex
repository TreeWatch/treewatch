Since this project was done using scrum, the typical phases as analysis, design and implementation as used in the waterfall methodology do not fit to us. But still the project could be broken down to analysis, design, implementation and documenting sprints. Each sprint took two weeks. 
\begin{description}
	\item[Analysis sprints] \hfill \\
	During the analysis sprints the project was set up and the user stories were written and validated by the customer. 
	\item[Design sprints] \hfill \\
	During the design sprints the first impression of a design was created using mockups. 
	\item[Implementation sprints] \hfill \\
	These were used to implement the functional prototype. The implementation sprints were planned as the most time consuming sprints.
	\item[Documenting sprint] \hfill \\
	At the end of the project there will be a documenting sprint, which is used only to document the everything what was done, so the VAA ICT Consultancy is able to work on the application after the project was finished by the SoFa.\ldots
\end{description}

After every sprint a sprint-meeting with the customer was scheduled to give an update to him, clarify any misunderstandings and receive possible new ideas from the customer. With this meeting the customer was able to take part at the development of the application and could influence it e.g. when he wanted another feature with a higher priority than others. The customer also could introduce totally new ideas which should be implemented first. This is the main strength of an agile methodology like scrum.