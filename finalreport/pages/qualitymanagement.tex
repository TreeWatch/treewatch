\section{Qualitymanagement}
This chapter describes the necessary information needed to manage the project quality from the project planning to the delivery
to the customer. Within this document the quality policies, procedures, roles, responsibilities and authorities are defined.\\
At the highest level Quality Management involves planning, doing, checking, and acting to improve project quality standards. Project Quality Management is therefore splitted into three categories: Quality Planning, Quality Assurance and Quality Control.
\subsection{Organization and Responsibilities}
\begin{tabular}{ | p{4cm} | p{3,5cm} |p{6,5cm} | }
	\hline
	\textbf{Name} & \textbf{Role} & \textbf{Quality Responsibility} \\ \hline
	Max van der Linden & Project Manager & External communication, Auditing \\ \hline
	Martijn Bonajo & Configuration Manager & Infrastructure, Source files, Software engineering documentation, Auditing \\ \hline
	Ron Gebauer & Scrum Master & Scrum planning, Auditing \\ \hline
	Rene Karoff & Quality Manager & Ensure use of quality guidelines, Auditing \\ \hline
	Jan Kerkenhoff & Main Engineer & Main Code-Master, Auditing \\
	\hline
\end{tabular}
\subsection{Quality Planning}
Since this project is about programming a mobile business application, it is highly important that the system has a minimum of bugs/errors. A good documentation is necessary too, not only because this application will be developed by another SoFa-group after our semester is over, but the customer should understand the system which is developed by us.
\subsubsection{Define Project Quality}
To ensure to quality of the written code it is mandatory to develop every module test-driven, therefore unit-testing is introduced. The written documents will be checked for its grammar, writing style and content.
\subsubsection{Measure Project Quality}
The code of the application will be tested. Therefore it is a goal to get at least 90\% code coverage, better 100\%. Each evaluation criterion of the written documents will be ranked from “--“ to “++” (“--“ too bad – “++” excellent).
\subsection{Quality Assurance}
Since the code is tested, this will show the group if the quality goal is achieved. Furthermore the written documents will be audited by at least one group member and the project quality manager.
\subsubsection{Analyze Project Quality}
The tool to measure to code coverage \textbf{(which still needs to be defined)}will show the programmer which code is still uncovered, and the writer of a document will receive feedback of the auditing persons, so he can improve on his writing too.
\subsubsection{Improve Project Quality}
Define proper requirements before starting programming the application.
\subsection{Quality Control}
At the end of the project each deliverable will be audited again, by every team member to ensure the project quality.
