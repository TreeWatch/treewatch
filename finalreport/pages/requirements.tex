\section{Requirements\label{sec:Requirements}}
The requirements definition was started by setting a meeting with Fleuren. Here Fleuren explained what their problem was and what they wanted to see as functionality of the app. Based on that meeting a set of user stories was created. These user stories were then grouped in epics. The epics were then sent back to Fleuren to have them prioritised and checked for completeness.

The checked and prioritised user stories were then put in a table to show which functionalities needed to be implemented first and which ones were optional. The table of user stories can be found below.

	\begin{longtable}{ p{.10\textwidth} p{.80\textwidth} }
		\textbf{Priority} & \textbf{User-story} \\
		\hline
		 1 & \underline{\textit{As a user}}, I want the ability to overlay the visualizations on top of the field. 
		
		\underline{\textit{As a user}}, I want to have my field represented graphically in the application. \\
		2 & \underline{\textit{As a user}}, I want the application to register as much data as possible, automatically.
		
		\underline{\textit{As a user}}, I want my GPS data entered into the system.
		
		\underline{\textit{As a user}}, I want my weather data entered into the system.
		
		\underline{\textit{As a user}}, I want my soil data gathered into the system.
		
		\underline{\textit{As a user}}, I want my system data entered into the system.
		
		\underline{\textit{As a user}}, I want to be able to take pictures of the trees.
		
		\underline{\textit{As a user}}, I want to be able to upload pictures and tag them to a tree inside a certain row/block.
		
		\underline{\textit{As a user}}, I want a system which visualizes the available data for the fields/blocks/rows/trees.
		
		\underline{\textit{As a user}}, I want to correlate the available data in different views so that I can see the relations between different datasets. \\
		3 & \underline{\textit{As a user}}, I want to digitize the kwekerij schrift (Nursery Script). \\
		4 & \underline{\textit{As a user}}, I want the application to contain different forms to transmit different information about the actual state of the trees on the field (e.g.~brown leaves, thin stems\ldots{}).
		
		\underline{\textit{As a user}}, I want to be able to see a chronological ordering of events that happened on trees in a certain row/block's history.
		
		\underline{\textit{As a user}}, I want to be able to see where and how long I worked on a field.
		
		\underline{\textit{As a user}}, I want to be able to specify what work I am doing on a field/block/row/tree.\\
		5 & \underline{\textit{As a user}}, I want the application to analyze the data of the trees and give hints and solutions for problems.\\
		6 & \underline{\textit{As a user}}, I want to have a system that manages my customers, their orders and the amount of trees that are growing or are fully grown for them.\\
		\caption{Priority \& User-story\label{tab:RequirementsAndUserStories}}
	\end{longtable}